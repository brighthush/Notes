\documentclass[UTF8]{ctexart}
\usepackage{graphicx}
\usepackage{amsmath}
\usepackage{bibentry,natbib}

\title{Paper Reading Note \\ Surveys}
\author{BrightHush}
\date{\today}

\begin{document}
\maketitle
\tableofcontents

\newcommand{\figref}[1]{\figurename~\ref{#1}}

\section{A Review of Relation Extraction}

\subsection{Abstract}
现如今的信息抽取、自然语言理解和信息检索都需要理解实体之间的语义关。这篇文章将会比较综合的
包括实体关系抽取方法的各个方面,其中重要的监督学习方法和非监督学习方法都会比较充分的进行
分析。另外我们会讨论扩展的高阶关系,对监督学习方法和半监督学习方法的评价都是采用常见的评比
数据集。最后,本文会介绍实体关系抽取方面重要的应用,问答系统和生物学文本挖掘。

\subsection{Supervised Methods}
首先我们把实体关系抽取问题当成一个分类问题,为了描述的简洁性,我们限制讨论的实体关系为二元
的,多元的关系抽取将会在接下来的部分进行讨论。对于给定的一个句子
$S=w_1, \dots , e_1, \dots, w_j, e_2, \dots$,其中$e_1,e_2$是实体,那么函数$f(.)$
可以描述如下:

\begin{equation}
f_R(T(S)) = 
\left\{
\begin{array}{ll}
+1, If \  e_1 \  and \  e_2 \  are \  related \  according \  to \ 
relation \ R
\\
-1, Otherwise
\end{array}
\right.
\end{equation}
\par
其中$T(S)$是从句子S中提取出来的特征,函数$f(\cdot)$用来判断实体是否存在关系。如果标记好的
关系正样本和负样本训练一个分类器,这个分类器可以是Perceptron, Voted Percetron or Support
Vector Machine。这些分类器使用的特征可以是通过文本分析得到的POS标记、依存关系,又或者是
结构化的一寸句法树。根据分类器输入的本质,实体关系抽取分类方法分成两类(1)基于特征的(2)核方法,
下面将会比较细节的讨论这两个方法。

\subsubsection{Feature based Methods}

\subsubsection{Kernel Methods}

\cite{EntityRelationExtraction}

\bibliographystyle{plain}
\bibliography{SurveyNote}

\end{document}
