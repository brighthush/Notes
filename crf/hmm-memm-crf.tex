\documentclass[UTF8]{ctexart}
\usepackage{graphicx}
\usepackage{amsmath}
\usepackage{bibentry,natbib}
\usepackage{fancyhdr}

\title{HMM-MEMM-CRF}
\author{Bright}
\date{\today}

\begin{document}
\maketitle
\tableofcontents

\pagestyle{fancy}
\cfoot{\thepage}

\newcommand{\figref}[1]{\figurename~\ref{#1}}
\section{Why this article}
在做序列标注的时候,我们首先想到要用的方法是CRF,因为这个是目前来说效果最好的方法。但是我们知道,
在CRF之前,HMM和MEMM同样可以用来做序列标注,为什么前两者的效果不如CRF呢?下面我们就从建模的
角度来看看为什么会这样。

\section{Hidden Markov Model}
在隐马尔科夫模型中,我们知道存在两种矩阵,一种是状态转移矩阵,我们用$A$表示;一种是发射矩阵,表示
从一个隐藏状态到一个观察值的概率,我们用$B$表示。
\par
下面的讨论中,我们约定几个变量的使用。输出变量用$x$表示,即$x = \{x_1, x_2, ..., x_T\}$,每一个
$x_i$的取值为$V=\{ v_1, v_2, ..., v_{|V|} \}$之一,状态变量用$z = \{z_1, z_2, ..., z_T\}$
表示,每个状态的取值为$S=\{s_1, s_2, ..., s_{|S|}\}$之一。
\par
HMM的建模过程中,存在两个假设:\\
a. 一阶马尔科夫假设,即$p(z_i|z_{i-1}, ..., z_1) = p(z_i|z_{i-1})$,当前状态的出现,仅仅依赖于前一个状态\\
b. 输出独立性假设,即$p(x_i|x_{i-1}, ..., x_1, z_i, ..., z_1) = p(x_i|z_i)$,即当前输出仅仅依赖于当前状态\\
\par
于是对于一个已知的状态序列和观察序列,我们很容易表示其联合概率如下
\begin{align}
p(x,z) &= p(x|z)p(z)
\\
&= p(x_1,..., x_T|z)p(z_1, ..., z_T)
\\
&= \Pi_{i=1}^{T}p(x_i|z_i) \Pi_{i=1}^{T} p(z_i|z_{i-1})
\\
&= \Pi_{i=1}^{T}p(x_i|z_i)p(z_i|z_{i-1})
\end{align}
\par
从上面的公式我们可以看到,在HMM中,我们建模的是联合概率,并且在序列中,每一时刻的观察值是相互独立的,并且仅仅
依赖于当前时刻的状态。因此这里的假设是非常强的,而且建模联合概率通常也不是我们所希望的,因此HMM的局限性比较大。

\section{Maximum Entropy Markov Model}
最大熵模型,想必大家都有所耳闻,MEMM只是将最大熵模型扩展到了Markov链上。最大熵模型的思想是,在满足约束条件的基础上,
我们保证熵最大,也就是我们并不对概率模型做任何的假设。
\par
在MEMM中,其建模的是条件概率,并且根据Markov假设的阶数不同,建模也会有所不同。下面以Trigram MEMMs为例,也就是说,
当前的输出,仅仅和前两个输出有关。
\begin{align}
p(y_1, ..., y_T | x_1, ..., x_T) &= 
\Pi_{i=1}^T p(y_i|x_1, ..., x_T, y_{i-2}, y_{i-1})
\\
p(y_i|x_1, ..., x_T, y_{i-2}, y_{i-1}) &= 
\frac{exp(\sum_{k=1}^K \theta_{k}f_k(x_1, ..., x_T, y_{i-2}, y_{i-1}, y_i)}
{\sum_y exp(\sum_{k=1}^K \theta_{k}f_k(x_1, ..., x_T, y_{i-2}, y_{i-1}, y)}
\end{align}
\par
从上面的公式我们可以看到,当前输出的$y_i$不仅依赖于所有的$x_i$,还依赖于前两阶的$(y_{i-2}, y_{i-1})$,那么
任意特征函数的输入就可以是$y_i$所依赖的内容。

\section{Conditional Random Field}

在CRF中建模的也是条件概率,这个时候$p(y|x)$并不是对单个的点进行连乘,CRF中$p(y|x)$表示的就是归一化的概率。
对于给定观察序列$\textbf{x}$,求状态序列$\textbf{y}$的条件概率如下:
\begin{align}
p(y|x) &= \frac{p(y, x)}{\sum_{y^{'}}p(y^{'}, x)}
\\
&= \frac{\Pi_{t=1}^T exp(\sum_{k=1}^K \theta_k f_k(y_t, y_{t-1}, x))}
{\sum_{y^{'}} \Pi_{t=1}^T exp(\sum_{k=1}^K \theta_k f_k(y_t^{'}, y_{t-1}^{'}, x)) }
\end{align}
\par
于是我们知道在CRF中的$p(y|x)$依赖于整体归一化的结果,而不是像MEMM中依赖于逐个的点进行连乘。所以CRF考虑的更加
像是全局最优解,因此CRF能解决像MEMM的转移偏置的问题,当然也比HMM更加的通用。

\section{References}
\begin{enumerate}
\item Hidden Markov Models Fundamentals, Daniel Ramage, CS229 Section Notes, Stanford University.
\item Chapter 8, MEMMS(Log-Linear Tagging Models)

\end{enumerate}


\end{document}
