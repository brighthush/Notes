\documentclass[UTF8]{ctexart}
\usepackage{graphicx}
\usepackage{amsmath}
\usepackage{bibentry,natbib}
\usepackage{fancyhdr}

\title{Conditional Random Fields}
\author{BrightHush}
\date{\today}

\begin{document}
\maketitle
\tableofcontents

\pagestyle{fancy}
\cfoot{\thepage}

\newcommand{\figref}[1]{\figurename~\ref{#1}}

\section{Model}

\subsection{Linear Chain CRFs}
在介绍 linear chain conditional random fields之前,首先介绍条件概率 $p(y|x)$,
这个条件可以可以通过HMM中的联合概率得到。重点在于条件分布实际上是使用特定特征函数的条件随机场。
\par
在这里我们首先重写HMM联合概率公式,当然是以更加通用的方式重写该公式,其中需要注意的是在HMM
中变量$\textbf{y}$表示状态序列,$\textbf{x}$表示观察序列,因此注意下面公式中各变量的含义及
公式本身表示的意义。
\begin{align}
p(\textbf{y}, \textbf{x}) = \frac{1}{Z} \Pi_{t=1}^T exp
\left( 
\sum_{i,j \in S}\theta_{ij}1_{y_t=i}1_{y_{t-1}=j} + 
\sum_{i \in S} \sum_{o \in O} \mu_{oi} 1_{y_t=i} 1_{x_t=o}
\right)
\end{align}
其中$\theta = ( \theta_{ij}, \mu_{oi} )$是该分布的实数参数,$Z$是归一化参数来保证
最后的概率之和为1。上式可以认为是HMM模型的一种泛化,通常的HMM可以当成该模型的一个特例。
当$\theta_{ij} = log p(y^{'}=i|y=j), \mu_{oi} = log p(x=o|y=i)$的时候,那么就是
通常我们所熟知的HMM模型了。

\par
通过引入\textit{feature functions},我们可以更加紧密的表示上面的公式。每一个特征函数的形式
为$y_k(y_t, y_{t-1}, x_t)$。为了能完整的表达上面的概率公式,需要有一个特征
$f_{ij}(y, y^{'}, x)=1_{y=i}1_{y^{'}=j}$表示每一个状态转移$(i, j)$,需要有一个特征
$f_{io}(y, y^{'}, x)=1_{y=i}1_{x=o}$表示状态-观察值对$(i, o)$。我们使用$f_k$表示特征
函数,$f_k$需要遍历所有的$f_{ij}$和所有的$f_{io}$。因此我们可以写作下面的方式:
\begin{align}
p(\textbf{y}, \textbf{x}) = \frac{1}{Z} \Pi_{t=1}^T exp
( \sum_{k=1}^K \theta_k f_k(y_t, y_{t-1}, x_t) )
\end{align}

\par
最后一步,对于给定观察序列$\textbf{x}$,求状态序列$\textbf{y}$的条件概率如下:
\begin{align}
p(y|x) &= \frac{p(y, x)}{\sum_{y^{'}}p(y^{'}, x)}
\\
&= \frac{\Pi_{t=1}^T exp(\sum_{k=1}^K \theta_k f_k(y_t, y_{t-1}, x_t))}
{\sum_{y^{'}} \Pi_{t=1}^T exp(\sum_{k=1}^K \theta_k f_k(y_t^{'}, y_{t-1}^{'}, x_t)) }
\end{align}
这是一个比较特别的 linear chain CRF,因为仅仅考虑了单独当前词对应的特征。而对于其他
很多的 linear chain CRF 会使用丰富的特征,例如当前词的前缀、后缀,或者是周围词的等等。
如果我们用更加通用的特征函数来替换当前词相关的特征函数,那么我们可以得到更加通用的 linear 
chain CRF:

\par
\textbf{Definition : } Let $Y, X$ be random vactors. Then a \textit{linear 
chain conditional random field} is a distribution $p(\textbf{y}|\textbf{x})$
that takes the form:
\begin{align}
p(\textbf{y} | \textbf{x}) = \frac{1}{Z(\textbf{x})} \Pi_{t=1}^T
exp( \sum_{k=1}^K \theta_k f_k(y_t, y_{t-1}, x_t) )
\end{align}
where $Z(x)$ is an instance specific normalization function
\begin{align}
Z(\textbf{x}) = \sum_{\textbf{y}} \Pi_{t=1}^T
exp( \sum_{k=1}^K \theta_k f_k(y_t, y_{t-1}, x_t) )
\end{align}

\par
通过以上我们可以看到,如果对联合概率$p(y,x)$按照HMM的方式进行分解,那么相应的条件概率$p(y|x)$
就是一个线性的crf模型。举例来说,在HMM中,通过状态i转到状态j会得到相同的分数$log p(y_t=j|y_{t-1}=i)$,
而没有参考输入。在CRF中,我们可以令(i,j)转移得到的分数依赖于当前的观察序列,通过增加一个特征
$1_{y_t=j}1_{y_{t-1}=i}1_{x_t=0}$。使用这种特征的CRF模型被广泛的应用在文本领域。

\par
最后,需要注意的是归一化常量$Z(x)$是对所有的可能的状态序列进行求和,这是一个指数级增长的数量。
然而,这个可以被 forward-backward 算法高效的计算。

\section{References}


\end{document}
